% Options for packages loaded elsewhere
\PassOptionsToPackage{unicode}{hyperref}
\PassOptionsToPackage{hyphens}{url}
%
\documentclass[
]{article}
\usepackage{amsmath,amssymb}
\usepackage{iftex}
\ifPDFTeX
  \usepackage[T1]{fontenc}
  \usepackage[utf8]{inputenc}
  \usepackage{textcomp} % provide euro and other symbols
\else % if luatex or xetex
  \usepackage{unicode-math} % this also loads fontspec
  \defaultfontfeatures{Scale=MatchLowercase}
  \defaultfontfeatures[\rmfamily]{Ligatures=TeX,Scale=1}
\fi
\usepackage{lmodern}
\ifPDFTeX\else
  % xetex/luatex font selection
\fi
% Use upquote if available, for straight quotes in verbatim environments
\IfFileExists{upquote.sty}{\usepackage{upquote}}{}
\IfFileExists{microtype.sty}{% use microtype if available
  \usepackage[]{microtype}
  \UseMicrotypeSet[protrusion]{basicmath} % disable protrusion for tt fonts
}{}
\makeatletter
\@ifundefined{KOMAClassName}{% if non-KOMA class
  \IfFileExists{parskip.sty}{%
    \usepackage{parskip}
  }{% else
    \setlength{\parindent}{0pt}
    \setlength{\parskip}{6pt plus 2pt minus 1pt}}
}{% if KOMA class
  \KOMAoptions{parskip=half}}
\makeatother
\usepackage{xcolor}
\usepackage[left=2cm,right=2cm,top=2.5cm,bottom=2.5cm]{geometry}
\usepackage{graphicx}
\makeatletter
\def\maxwidth{\ifdim\Gin@nat@width>\linewidth\linewidth\else\Gin@nat@width\fi}
\def\maxheight{\ifdim\Gin@nat@height>\textheight\textheight\else\Gin@nat@height\fi}
\makeatother
% Scale images if necessary, so that they will not overflow the page
% margins by default, and it is still possible to overwrite the defaults
% using explicit options in \includegraphics[width, height, ...]{}
\setkeys{Gin}{width=\maxwidth,height=\maxheight,keepaspectratio}
% Set default figure placement to htbp
\makeatletter
\def\fps@figure{htbp}
\makeatother
\setlength{\emergencystretch}{3em} % prevent overfull lines
\providecommand{\tightlist}{%
  \setlength{\itemsep}{0pt}\setlength{\parskip}{0pt}}
\setcounter{secnumdepth}{-\maxdimen} % remove section numbering
\ifLuaTeX
\usepackage[bidi=basic]{babel}
\else
\usepackage[bidi=default]{babel}
\fi
\babelprovide[main,import]{french}
% get rid of language-specific shorthands (see #6817):
\let\LanguageShortHands\languageshorthands
\def\languageshorthands#1{}
\ifLuaTeX
  \usepackage{selnolig}  % disable illegal ligatures
\fi
\IfFileExists{bookmark.sty}{\usepackage{bookmark}}{\usepackage{hyperref}}
\IfFileExists{xurl.sty}{\usepackage{xurl}}{} % add URL line breaks if available
\urlstyle{same}
\hypersetup{
  pdftitle={DM 1 : Manipulations élémentaires autour de l'inertie},
  pdfauthor={EL MAZZOUJI Wahel; GILLET Louison},
  pdflang={true},
  hidelinks,
  pdfcreator={LaTeX via pandoc}}

\title{DM 1 : Manipulations élémentaires autour de l'inertie}
\author{EL MAZZOUJI Wahel \and GILLET Louison}
\date{2024/2025}

\begin{document}
\maketitle

\begin{figure}[h!]
    \centering
    \includegraphics[width=0.5\linewidth]{ssd.png}
\end{figure}

\newpage
\tableofcontents
\cleardoublepage

\hypertarget{introduction}{%
\section{Introduction}\label{introduction}}

Dans le cadre de notre étude, nous avons accès à un jeu de données riche
qui examine la diversité de 27 espèces d'arbres au sein de 1000
parcelles forestières. Cette analyse vise à explorer la variabilité des
densités de peuplement de ces espèces dans le contexte particulier de la
forêt du bassin du Congo. Le jeu de données se compose de 30 variables
quantitatives, dont les principales incluent le comptage des individus
pour chaque espèce, la superficie de chaque parcelle ainsi que deux
variables supplémentaires relatives au type forestier et au type
géologique. À cela s'ajoute une variable qualitative, identifiée par un
``code'', qui permet d'apporter des informations contextuelles sur
chaque parcelle. Cette étude permettra d'éclairer les dynamiques
écologiques en jeu et d'approfondir notre compréhension des interactions
entre les espèces arborées et leur environnement.

\hypertarget{partie-1}{%
\section{Partie 1}\label{partie-1}}

Nous cherchons à calculer la densité de peuplement de chaque espèce par
unité de surface. Pour chaque parcelle, la densité est donnée par :

\[
D_{ij} = \frac{N_{ij}}{S_{j}}
\]

où \(D_{ij}\) est la densité pour l'espèce \(i\) dans la parcelle \(j\),
\(N_{ij}\) est le nombre d'individus de l'espèce \(i\) dans la parcelle
\(j\) et \(S_{j}\) est la surface de la parcelle \(j\).

Nous utilisons des densités plutôt que des comptages car cela permet de
normaliser les données par rapport à la taille de la parcelle, ce qui
rend les comparaisons entre les parcelles équitables.

\end{document}
